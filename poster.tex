\documentclass[20pt,a1paper,landscape]{tikzposter}

% required packages
\usepackage{amsmath}
\usepackage{amsfonts}
\usepackage{amsthm}
\usepackage{tikz}
\usepackage{graphicx}
\usepackage{natbib}
\usepackage{url}

% compact bibliography
\renewcommand{\bibsection}{}
\setlength{\bibsep}{0pt plus 0.3ex}

% tiks fun
\usetikzlibrary{shapes,decorations,arrows,calc,arrows.meta,fit,positioning}
\tikzset{
    -Latex,auto,node distance =1 cm and 1 cm,semithick,
    state/.style ={ellipse, draw, minimum width = 1 cm},
    point/.style = {circle, draw, inner sep=0.1cm,fill,node contents={}},
    bidirected/.style={Latex-Latex,dashed},
    el/.style = {inner sep=2pt, align=left, sloped}
}

% set up new enumerate style for assumptions
\renewcommand{\labelenumi}{(A\arabic{enumi})}

% math macros
\newcommand{\D}{\mathcal{D}}
\newcommand{\E}{\mathbb{E}}
\newcommand{\F}{\mathcal{F}}
\newcommand{\M}{\mathcal{M}}
\newcommand{\R}{\mathbb{R}}
\newcommand{\X}{\mathcal{X}}
\newcommand{\logit}{\text{logit}}
\newcommand{\like}{\mathcal{L}}
\renewcommand{\P}{\mathbb{P}}
\newcommand{\I}{\mathbbm{I}}
\newcommand{\1}{\mathbbm{1}}
\newcommand{\indep}{\mbox{$\perp\!\!\!\perp$}}
\DeclareMathOperator*{\argmin}{\arg\!\min}
\DeclareMathOperator*{\argmax}{\arg\!\max}

%% Available themes: see also
%% https://bitbucket.org/surmann/tikzposter/downloads/themes.pdf
\usetheme{Default}
\colorlet{backgroundcolor}{white}

% title information
\title{Identifying Direct Causal Effects Under Unmeasured Confounding}
\author{Philippe Boileau$^{\ast 1}$, Nima S. Hejazi$^{\ast 2}$,
  Ivana Malenica$^{\ast 1}$, Sandrine Dudoit$^1$, Mark J. van der Laan$^1$}
\institute{$^1$University of California, Berkeley; $^2$Weill Cornell Medicine}
%\titlegraphic{}

% dictate default block options
\newcommand{\myblock}[2]{\block[titleinnersep=2mm, linewidth=0.5mm]{#1}{#2}}
\newcommand{\mysmallblock}[2]{\block[titleinnersep=1mm, linewidth=1mm, bodyinnersep=5mm, roundedcorners=12, ]{{\small #1}}{{\tiny#2\par}}}

\begin{document}

\maketitle[width = 26in]
\node[anchor=west] at (TP@title.west) {\includegraphics[width=5cm]{logos/cal}};
\node[anchor=east] at (TP@title.east) {\includegraphics[scale=0.7]{logos/wcm}};


\begin{columns}

  \column{0.333}

  \myblock{Introduction \& Motivations}{
    \begin{itemize}
      \itemsep2pt
        \item Developing \textit{mechanistic} understandings of causal effects
          is a ubiquitous goal across scientific disciplines.
        \item The natural direct and indirect effects are common target causal
          parameters --- both identifiable without structural assumptions.
        \item Identification assumes absence of unmeasured confounders of
          exposure--mediator, mediator--outcome, and exposure--outcome
          pathways --- but this is \textit{not} completely necessary.
        \item The natural direct and indirect effects arise from a
          decomposition of the average treatment effect (ATE), defined by a
          \vspace{-10pt}
          \begin{itemize}
            \itemsep1pt
            \item \textit{joint} intervention on exposure ($A$) and mediators
              ($Z$), where
            \item natural indirect effect (NIE) captures the portion of the ATE
              that passes only through the mediators, while the
            \item natural direct effect (NDE) captures the remainder of the
              ATE, that is, through all paths excluding $Z$.
          \end{itemize}
    \end{itemize}
  }

  \myblock{The Statistical Problem}{
    Full (unobserved) data is $X = (W, V, A, Z, Y)$, and we observe $O = (W, A,
    Z, Y)$. Exposure--mediator confounder $V$ \textbf{not} part of $O$. \\

    \vspace{-10pt}
    Let $O \sim P_0 \in \M$, with $\M$ being the nonparametric
    \textit{statistical model}. The likelihood of the observed data under $\M$
    is:
    \begin{equation*}
      p(o) = \prod_{i=1}^{n} q_Y(y_i \mid w_i, a_i, z_i) q_Z(z_i \mid w_i, a_i)
        g_A(a_i \mid w_i) q_W(w_i) \ .
    \end{equation*}

    \vspace{-10pt}
    The average treatment effect may be decomposed as
    \begin{align*}
      \Psi_{\text{ATE}}(P_{X,0})
       =& \E_{X,0}[Y(1) - Y(0)] \\
       =& \underbrace{\E_{X,0}[Y(1, Z(1)) - Y(1, Z(0))]}_{\text{NIE}} \\
       &+ \underbrace{\E_{X,0}[Y(1, Z(0)) - Y(0, Z(0))]}_{\text{NDE}} \ ,
    \end{align*}
    where $Y(1, Z(0))$ arises from a joint intervention on the exposure and
    mediators, setting them to \textit{incompatible} values. The NDE is
    \begin{align*}
      \Psi_{\text{NDE}}(P_{X,0}) &= \int_{\mathcal{W}, \mathcal{Z}}
        \E[Y(1,z) - Y(0,z) \mid W=w] \\
        & \qquad\qquad\qquad p_Z(z \mid A=0,w)p_W(w)~d\mu(z)~d\nu(w) \ .
    \end{align*}

    \vspace{-1em}
  }

  \column{0.334}

  \myblock{Causal Identification}{

    \centering
    \begin{tikzpicture}
      \node[state] (1) at (0,7.0) {$W$};
      \node[state] (2) at (0,4) {$Z$};
      \node[state,rectangle] (3) at (0,0) {$V$};
      \node[state] (4) at (-6,2) {$A$};
      \node[state] (5) at (6,2) {$Y$};

      \path (1) edge (2);
      \path (1) edge (5);
      \path (1) edge (4);

      \path (4) edge[densely dotted] (5);
      \path (4) edge[dashed] (2);
      \path (2) edge[dashed] (5);

      \path (3) edge (4);
      \path (3) edge (2);

      \node[draw=red,loosely dotted,fit=(3) (5), inner sep=0.2cm,line width=0.05cm] (machine) {};
    \end{tikzpicture}

    \begin{enumerate}
      \item No unmeasured endogenous pathways:\newline
        $f_Y(Z, A, W, V, U_Y) \equiv f_Y(Z, A, W, U_Y)$
      \item Conditional expectation equivalence:\newline
        $\E(Y \mid Z,A=1,W,V) \equiv \E(Y \mid Z,A=1,W)$
    \end{enumerate}

    \innerblock[]{Theorem}{
      Under assumptions A1 and A2, $\Psi_{\text{NDE}}(P_{X,0})$ is
      identified by
      \begin{align*}
        \Psi(P_0)
        &= \E_{P_0}\E_{P_0} \{\E_{P_0}(Y \mid W,A=1,Z) \\
        &\qquad\qquad\qquad - \E_{P_0}(Y \mid W,A=0,Z) \mid A=0,W\} \ .
      \end{align*}
    }
  }

  \myblock{Estimation \& Inference}{

    \textbf{Existing estimation and testing approaches are compatible with
    this relaxed identification strategy.} For example, one could use the
    \begin{itemize}
      \itemsep0pt
      \item targeted maximum likelihood estimator of~\citet{zheng2012}, or the
      \item one-step estimator of~\citet{tchetgen2011}.
    \end{itemize}
    Both are \textit{multiple-robust} (being based on the efficient influence
    function), \textit{asymptotically linear} under fairly non-restrictive
    assumptions, and compatible with \textit{cross-fitting} of nuisance
    estimators.

    \vspace{0.5em}
    Both estimators are implemented in the \texttt{medoutcon} \texttt{R}
    package (check it out at \url{https://github.com/nhejazi/medoutcon}).
  }

  \mysmallblock{Funding}{
    PB's gratefully acknowledges the support of the FRQNT and NSERC. NSH's work
    was supported by NSF DMS 2102840.
  }

  \column{0.333}

  \myblock{Numerical Experiments}{

    We consider the following data-generating process:
    \begin{align*}\label{eq:nde-sim-dgp}
      W_1 & \sim \text{Unif}(-1, 1); \: W_2, V \sim \text{Norm}(0, 1) \\
      A|W, V & \sim \text{Bern}
      \left(\left(1 + \exp\{-W_1-W_2-V\}\right)^{-1}\right) \\
      Z|A, W, V & \sim \text{Bern}
      \left(\left(1 + \exp\{-W_1-W_2- \gamma V - 3A\}\right)^{-1}\right) \\
      Y| Z, A, W, V & \sim \text{Norm}(3A + W_1 + W_2 + Z, 1) \ .
    \end{align*}
    \centering

    \vspace{-2em}

    \begin{tikzfigure}
      \includegraphics[width=0.295\textwidth]{figs/impact-unmeasured-conf-eff-size-sim.jpeg}
    \end{tikzfigure}
    \vspace{-2em}
  }

  \myblock{Conclusions}{
    \vspace{-1em}
    \begin{itemize}
      \itemsep1pt
        \item The NDE is identifiable under unmeasured exposure--mediator
          confounding by a \textit{well-studied} statistical functional!
        \item The NDE has recently been used to study the \textit{proportion of
          vaccine effect mediated} through candidate immune correlates.
          \begin{itemize}
            \itemsep0pt
            \item Our results strengthen ``out-of-the-box'' uses of the NDE.
            \item Exposure--mediator confounders in vaccine studies: prior
              infection history, immunocompromised status, genetics.
          \end{itemize}
    \end{itemize}
    \vspace{-1.5em}
  }

  \mysmallblock{References}{
    \vspace{-3em}
    \bibliographystyle{unsrtnat}
    \bibliography{refs}
    \vspace{-1em}
  }

  \begin{subcolumns}
    \subcolumn{0.65}
    \block[linewidth=0mm, bodyinnersep=5mm, roundedcorners=12]{}{
      \vspace{-0.5em}
      {\small \textit{$^\ast$ indicates shared first-authorship, ordered
      alphabetically.}}
      \vspace{-1em}
    }
  \end{subcolumns}


\end{columns}

\end{document}
